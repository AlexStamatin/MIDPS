\section{Efectuarea lucrarii de laborator}

\subsection{Sarcinile propuse}
\begin{itemize}
	\item Realizarea unui website personal unde vor fi afisate wallpaper-uri de inalta calitate din diferite categorii
	\item Pastrarea tuturor imaginilor afisate pe site intr-o baza de date pentru simplificarea adaugarii si stergerii acestora
\end{itemize}


\subsection{Analiza lucrarii de laborator}

Link la repozitoriu: https://github.com/AlexStamatin/MIDPS
	
	In cadrul acestei lucrari de laborator am elaborat un website pentru publicarea wallpaper-urilor din diferite categorii. Pentru realizarea acestui scop a fost utilizat framework-ul Django 1.11 pentru Python 3.6. De asemenea au fost utilizate mai multe module, cum ar fi Crispy forms pentru aspectul grafic al campurilor de introducere a textului si butoanelor si Django Registration Redux pentru realizarea sistemului de inregistrare pe website.

\begin{itemize}
	\item Interfata site-ului
	

	Toate paginile site-ului contin o bara de navigare NavBar realizata cu ajutorul framework-ului Bootstrap pentru accesarea mai usoara a functionalitatilor site-ului.
	Prima pagina contine un banner cu o descriere scurta a scopului prezentului website si optiuni pentru logare/inregistrare. De asemenea sunt prezentate categoriile de imagini disponibile.	
		\begin{figure}[h!]
			\centering
 			 \includegraphics[scale=0.5]{"FrontPage"}
 			 \caption{Pagina principala}
 			 \label{fig:Pagina principala}
		\end{figure}	
	\newpage
	In cadrul fiecarei categorii sunt prezentate imagini insotite de titlu si data incarcarii
	
	\begin{figure}[h!]
			\centering
 			 \includegraphics[scale=0.5]{"AnimePage"}
 			 \caption{Pagina unei categorii de imagini}
 			 \label{fig:AnimePage}
		\end{figure}
\newpage
	Este prezenta optiunea de feedback din partea utilizatorilor in caz de sugestii sau plangeri cu privire la drepturile de autor
	
	\begin{figure}[h!]
			\centering
 			 \includegraphics[scale=0.5]{"ContactPage"}
 			 \caption{Pagina pentru feedback}
 			 \label{fig:Calcui}
		\end{figure}
		
	Pentru o navigare mai comoda a site-ului utilizatorii au optiunea de a se inregistra
	\newpage
	\begin{figure}[h!]
			\centering
 			 \includegraphics[scale=0.5]{"RegistrationPage"}
 			 \caption{Pagina de inregistrare}
 			 \label{fig:Calcui}
		\end{figure}
\newpage
	\item Functionalitatea website-ului
	
	Toate imaginile de pe site sunt pastrate intr-o baza de date. Acestea au un titlu, o anumita categorie si data de incarcare.Imaginile pot fi usor adaugate sau scoase la dorinta administratorului

	\begin{figure}[h!]
			\centering
 			 \includegraphics[scale=0.5]{"Images"}
 			 \caption{Baza de date a imaginilor}
 			 \label{fig:Calcui}
		\end{figure}
		
	Datele despre utilizatori de asemenea se pastreaza intr-o baza de date. Administratorul poate adauga utilizatori noi, activa conturile acestora sau ii poate sterge din baza.


	\begin{figure}[h!]
			\centering
 			 \includegraphics[scale=0.5]{"Users"}
 			 \caption{Baza de date a utilizatorilor}
 			 \label{fig:Calcui}
		\end{figure}


\end{itemize}
