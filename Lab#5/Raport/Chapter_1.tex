\section{Efectuarea lucrarii de laborator}

\subsection{Sarcinile propuse}
\begin{itemize}
	\item Dezvoltarea unei aplicatii pentru management-ul eficient al contactelor
	\item Pastrarea tuturor contactelor intr-o baza de date cu optiuni de adaugare, stergere si editare a acestora
\end{itemize}


\subsection{Analiza lucrarii de laborator}

Link la repozitoriu: https://github.com/artiomnichifor/midps
		
	In cadrul acestei lucrari de laborator am elaborat o aplicatie pentru management-ul contactelor la telefonul mobil. Pentru realizarea acestui scop am utilizat mediul de dezvoltare a aplicatiilor Android Studio. Pentru pastrarea contactelor am utilizat o baza de date Sqlite si am utilizat functionalitati aditionale din SqlHelper.

\begin{itemize}
	\item Interfata aplicatiei
	

Interfata cu utilizatorul a fost creata cu ajutorul instrumentelor disponibile in Android Studio. Au fost utilizate campuri pentru introducerea textului si butoane pentru navigare
		\begin{figure}[h!]
			\centering
 			 \includegraphics[scale=1]{"firstpage"}
 			 \caption{Adaugarea unui contact in lista}
 			 \label{fig:Pagina principala}
		\end{figure}	
	\newpage
Informatia despre fiecare contact poate fi vuzializata in cadrul unei liste
	
	\begin{figure}[h!]
			\centering
 			 \includegraphics[scale=1]{"2nd page"}
 			 \caption{Lista de contacte}
 			 \label{fig:AnimePage}
		\end{figure}

\newpage
	\item Functionalitatea aplicatiei
	
	Toate contactele din cadrul aplicatiei sunt pastrate intr-o baza de date SQLite. 
	\begin{figure}[h!]
			\centering
 			 \includegraphics[scale=1]{"contacts"}
 			 \caption{Formatul de pastrare a contactelor}
 			 \label{fig:Calcui}
		\end{figure}
		
Pentru operarea cu acestea au fost utilizate functii din SQLiteDataBase si SQLiteOpenHelper


\lstinputlisting[language=C++, caption={Utilizarea bazei de date},firstline=15,lastline=28]{DatabaseHandler.java}

In cadrul lucrului in echipa, fiecare membru a muncit pe un branch iar la sfarsit am realizat merge pe branch-ul master.

	\begin{figure}[h!]
			\centering
 			 \includegraphics[scale=1]{"merge"}
 			 \caption{Utilizarea comenzii merge}
 			 \label{fig:Calcui}
		\end{figure}


\end{itemize}
