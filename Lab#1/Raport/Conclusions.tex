\section*{Concluzie}
\phantomsection


In cadrul acestei lucrari de laborator am studiat sistemul de control al versiunilor Git. Am determinat ca acesta permite uramrirea istoriei modificarilor suportate de un fisier.

Utilizarea Git faciliteaza munca inginerilor in domeniul software oferindule un sir de posibilitati:
\begin{itemize}

\item Posibilitatea de a reveni la o versiune mai veche functionala a codului in cazul comiterii unor greseli
\item Compararea diferitor versiuni a codului propriu
\item Urmarirea dezvoltarii unui program pe parcursul unei perioade de timp 
\item Experimentarea cu noi functionalitati fara a interfera cu versiunile de baza a codului

\end{itemize}

De asemenea utilizarea unui VCS este absolut indispensabila in cazul lucrului in echipa asupra unui proiect, deoarece:

\begin{itemize}
\item Permite vizualizarea informatiei despre cine a lucrat asupra unui proiect si ce schimbari a introdus
\item Permite mai multor persoane sa lucreze in paralel asupra diferitor parti a unui proiect avand posibilitatea de a combina ulterior modificarile facute la program
\item Ofera posibilitatea de a determina ce secventa de cod a dus la aparitia bug-urilor in program
\end{itemize}
In urma efectuarii lucrarii ne-am familiarizat cu comenzile de baza ale sistemului Git si am obtinut abilitati de utilizare a diferitor functionalitati ale acestuita. Am reusit crearea mai multor branch-uri, combinarea acestora si rezolvarea conflictelor ce au aparut pe parcurs. Am avut posibilitatea de a ne intoarce la versiuni mai vechi a fisierelor in cazul unor schimbari nedorite sau pierderii datelor. 
Am ajuns la concluzia ca cunoasterea sistemului Git poate spori considerabil performantele programatorilor.


\clearpage