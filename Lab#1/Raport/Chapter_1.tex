\section{Efectuarea lucrarii de laborator}

\subsection{Sarcinile propuse}
\begin{itemize}
	\item Initializarea unui nou repozitoriu
	\item Generarea si adaugarea cheilor SSH
	\item Configurarea VCS 
	\item Crearea branch-urilor (cel putin 2)
	\item Cel putin 1 commit pe fiecare branch
	\item Setarea unui branch to track a remote origin
	\item Resetarea unui branch la commit-ul anterior
	\item Salvarea temporara a schimbarilor fara commit imediat
	\item Utilizarea fisierului .gitignore
	\item Merge la 2 branch-uri
	\item Rezolvarea conflictelor
	\item Utilizarea tag-urilor
\end{itemize}


\subsection{Analiza lucrarii de laborator}

Link la repozitoriu: https://github.com/AlexStamatin/MIDPS

\begin{itemize}
	\item Initializarea unui nou repozitoriu
		
		Aceasta sarcina poate fi indeplinita atat online utilizand site-ul github cat si utilizand comanda \textit{git init}
		In cazul dat repozitoriul nou a fost creat accesand pagina personala de pe github cu ajutorul optiunii "New Repository" din compartimentul "Your repositories".
		
		\begin{figure}[h]
			\centering
 			 \includegraphics[scale=0.75]{"task 1 create repo"}
 			 \caption{Crearea repozitoriului}
 			 \label{fig:create_repo}
		\end{figure}
\newpage
Repozitoriul a fost clonat pe masina locala utilizand comanda \textit{git clone}.

		\begin{figure}[h!]
			\centering
 			 \includegraphics[scale=0.75]{"task 1 cloning"}
 			 \caption{Clonarea pe masina locala}
 			 \label{fig:clone_repo}
		\end{figure}
		
\item Generarea si adaugarea cheilor SSH

Utilizand protocolul SSH este posibila autentificarea la servere si servicii aflate la distanta. Cu ajutorul cheilor SSH este posibila conectarea la github fara a fi necesara introducerea username-ului si parolei la fiecare vizita.
Cheile SSH pot fi generate cu ajutorul comenzii \textit{ssh-keygen}. Cheile publice existente pot fi afisate cu ajutorul comenzii \textit{cat $\sim$ $/.ssh/.ssh/id\_rsa.pub$}

	\begin{figure}[h!]
			\centering
 			 \includegraphics[scale=0.75]{"task 2 SSH"}
 			 \caption{Cheia publica SSH}
 			 \label{fig:SSH_key}
		\end{figure}

\item Configurarea VCS

Configurarile git de baza pot fi modificate cu ajutorul comenzii \textit{git config --global user.name "name"} si \textit{git config --global user.email "email"}

\begin{figure}[h!]
			\centering
 			 \includegraphics[scale=0.75]{"task 3 gitconfig"}
 			 \caption{Configurarea unor parametri}
 			 \label{fig:gitconfig}
		\end{figure}
		
Fisiere noi spre indexare pot fi adaugate cu ajutorul comenzii \textit{git add}. Pentru a inregistra toate schimbarile in comparatie cu fisierele de pe git se utilizeaza comanda \textit{git commit -m "Commit comment"}. Pentru a incarca fisierele indexate pe git utilizam comanda \textit{git push}

\begin{figure}[h!]
			\centering
 			 \includegraphics[scale=0.75]{"task 3 commit"}
 			 \caption{Adaugarea fisierelor la repozitoriu}
 			 \label{fig:commit}
		\end{figure}
		
\item Crearea branch-urilor

Pentru a crea un branch este necesara utilizarea comenzii \textit{git branch "name"} Comanda \textit{git branch} ne afiseaza branch-urile. Pentru a crea un branch nou si a face switch la el se utilizeaza comanda \textit{git branch -b "name"}

\begin{figure}[h!]
			\centering
 			 \includegraphics[scale=0.75]{"task 4 create branch"}
 			 \caption{Crearea branch-urilor}
 			 \label{fig:create_branch}
		\end{figure}
\newpage
\item Crearea commit-urilor de pe fiecare branch

Initial facem switch la branch-ul de pe care dorim sa facem commit cu ajutorul comenzii \textit{git checkout "name"}. Apoi cu \textit{git add} adaugam fisierele spre indexare.

\begin{figure}[h!]
			\centering
 			 \includegraphics[scale=0.75]{"task 5 NBranch1"}
 			 \caption{Commit in NBranch1}
 			 \label{fig:NBranch1}
		\end{figure}

\begin{figure}[h!]
			\centering
 			 \includegraphics[scale=0.75]{"task 5 NBranch2"}
 			 \caption{Commit in NBranch2}
 			 \label{fig:NBranch2}
		\end{figure}
\newpage
\item Setarea unui branch to track a remote origin

A fost creat un nou branch trbranch to track remote origin utilizand comanda \textit{git branch --track "name" origin/master}. Ulterior a fost creat si adaugat un commit in trbranch cu comanda \textit{git push origin "name"}

\begin{figure}[h!]
			\centering
 			 \includegraphics[scale=0.75]{"task 6 track"}
 			 \caption{Track remote origin}
 			 \label{fig:track}
		\end{figure}

\begin{figure}[h!]
			\centering
 			 \includegraphics[scale=0.75]{"task 6 push"}
 			 \caption{Push to trbranch}
 			 \label{fig:push}
		\end{figure}

\item Resetarea unui branch la commit-ul anterior

Cu ajutorul comenzii \textit{git log} identificam commit-ul la care dorim sa resetam branch-ul

\begin{figure}[h!]
			\centering
 			 \includegraphics[scale=0.75]{"task 7 log"}
 			 \caption{Identificarea commit-ului}
 			 \label{fig:log}
		\end{figure}

Apoi cu ajutorul \textit{git reset "mode" "commit"} se reseteaza head-ul branch-ului curent la "commit". Poate fi utilizat unul dintre modurile soft, mixed(optiunea default), hard, merge sau keep.

\item Salvarea temporara a schimbarilor fara commit imediat

Uneori este necesar ca in timp ce se lucreaza asupra unei parti din proiect sa se faca switch la un alt branch dar fara a face un commit a modificarilor curente. Acest lucru poate fi realizat cu ajutorul comenzii \textit{git stash}

\begin{figure}[h!]
			\centering
 			 \includegraphics[scale=0.75]{"task 8 stash"}
 			 \caption{Utilizarea git stash}
 			 \label{fig:stash}
		\end{figure}

Lista stash-urilor pastrate poate fi vizualizata cu comanda \textit{git stash list}

\begin{figure}[h!]
			\centering
 			 \includegraphics[scale=0.75]{"task 8 stashlist"}
 			 \caption{Vizualizarea stash-urilor curente}
 			 \label{fig:stashlist}
		\end{figure}
		
Stash-urile pastrate pot fi aplicate cu ajutorul \textit{git stash apply}
\newpage
	\item Utilizarea fisierului .gitignore
	
Fisierul .gitignore este utilizat pentru a ne asigura ca anumite fisiere nu vor fi indexate si vor fi ignorate de catre git. Acesta nu afecteaza fisierele care deja sunt indexate.

Printre fisierele ce se recomanda a fi ignorate se numara fisiere auziliare LaTeX, imagini e.t.c.

\begin{figure}[h!]
			\centering
 			 \includegraphics[scale=0.75]{"task 9 gitignore"}
 			 \caption{Fisierul .gitignore}
 			 \label{fig:gitignore}
		\end{figure}
\newpage
\item Merge la 2 branch-uri si rezolvarea conflictelor

Mai multe branch-uri pot fi adaugate la un singur branch cu ajutorul comenzii \textit{git merge}. Dar pot aparea conflicte din cauza diferentelor dintre continuturile fisierelor indexate in aceste branch-uri.

\begin{figure}[h!]
			\centering
 			 \includegraphics[scale=0.75]{"task 9 merge"}
 			 \caption{Conflictul aparut in urma comenzii git merge}
 			 \label{fig:merge}
		\end{figure}

Conflictul poate fi rezolvat manual editand fisierele ce il cauzeaza.

\begin{figure}[h!]
			\centering
 			 \includegraphics[scale=0.75]{"task 9 conflict"}
 			 \caption{Rezolvarea conflictului}
 			 \label{fig:conflict}
		\end{figure}

Dupa rezolvarea conflictului se paote efectua merge-ul

\begin{figure}[h!]
			\centering
 			 \includegraphics[scale=0.75]{"task 9 solved"}
 			 \caption{Commit dupa rezolvarea conflictului}
 			 \label{fig:solved}
		\end{figure}
\newpage
\item Utilizarea tag-urilor

Tag-urile sunt utilizate pentru a marca commit-uri importante. Exista doua tipuri de tag-uri: lightweight si annotated. Un tag lightweight este ca un branch care nu se schimba, este un pointer la un commit specific. Pentru a utiliza acest tip de tag se apeleaza comanda \textit{git tag "tagname"}. Tag-urile annotated sunt pastrate ca obiecte in baza de date Git. Pentru a crea asa un tag se utilizeaza comanda \textit{git tag -am "mesaj" "tagname"}

\begin{figure}[h!]
			\centering
 			 \includegraphics[scale=0.75]{"task fin lw"}
 			 \caption{Crearea unui tag lightweight}
 			 \label{fig:solved}
		\end{figure}

\begin{figure}[h!]
			\centering
 			 \includegraphics[scale=0.75]{"task fin lw"}
 			 \caption{Crearea unui tag annotated si indexarea tagurilor}
 			 \label{fig:solved}
		\end{figure}

\end{itemize}

