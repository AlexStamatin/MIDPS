\section{Efectuarea lucrarii de laborator}

\subsection{Sarcinile propuse}
\begin{itemize}
	\item Initializarea unui nou repozitoriu
	\item Generarea si adaugarea cheilor SSH
	\item Configurarea VCS 
	\item Crearea branch-urilor (cel putin 2)
	\item Cel putin 1 commit pe fiecare branch
	\item Setarea unui branch to track a remote origin
	\item Resetarea unui branch la commit-ul anterior
	\item Salvarea temporara a schimbarilor fara commit imediat
	\item Utilizarea fisierului .gitignore
	\item Merge la 2 branch-uri
	\item Rezolvarea conflictelor
\end{itemize}


\subsection{Analiza lucrarii de laborator}

Link la repozitoriu: https://github.com/AlexStamatin/MIDPS

\begin{itemize}
	\item Initializarea unui nou repozitoriu
		
		Aceasta sarcina poate fi indeplinita atat online utilizand site-ul github cat si utilizand comanda \textit{git init}
		In cazul dat repozitoriul nou a fost creat accesand pagina personala de pe github cu ajutorul optiunii "New Repository" din compartimentul "Your repositories".
		
		\begin{figure}[h]
			\centering
 			 \includegraphics[scale=0.75]{"task 1 create repo"}
 			 \caption{Crearea repozitoriului}
 			 \label{fig:create_repo}
		\end{figure}
\newpage
Repozitoriul a fost clonat pe masina locala utilizand comanda \textit{git clone}.

		\begin{figure}[h!]
			\centering
 			 \includegraphics[scale=0.75]{"task 1 cloning"}
 			 \caption{Clonarea pe masina locala}
 			 \label{fig:clone_repo}
		\end{figure}
		
\item Generarea si adaugarea cheilor SSH

Utilizand protocolul SSH este posibila autentificarea la servere si servicii aflate la distanta. Cu ajutorul cheilor SSH este posibila conectarea la github fara a fi necesara introducerea username-ului si parolei la fiecare vizita.
Cheile SSH pot fi generate cu ajutorul comenzii \textit{ssh-keygen}. Cheile publice existente pot fi afisate cu ajutorul comenzii \textit{cat $~/.ssh/.ssh/id_rsa.pub\sim$}

	\begin{figure}[h!]
			\centering
 			 \includegraphics[scale=0.75]{"task 2 SSH"}
 			 \caption{Cheia publica SSH}
 			 \label{fig:clone_repo}
		\end{figure}

\item Configurarea VCS

Configurarile git de baza pot fi modificate cu ajutorul comenzii \textit{git config --global user.name <name>} si \textit{git config --global user.email <email>}

\begin{figure}[h!]
			\centering
 			 \includegraphics[scale=0.75]{"task 3 gitconfig"}
 			 \caption{Configurarea unor parametri}
 			 \label{fig:clone_repo}
		\end{figure}
		
Fisiere noi spre indexare pot fi adaugate cu ajutorul comenzii \textit{git add}. Pentru a inregistra toate schimbarile in comparatie cu fisierele de pe git se utilizeaza comanda \textit{git commit -m "Commit comment"}. Pentru a incarca fisierele indexate pe git utilizam comanda \textit{git push}

\begin{figure}[h!]
			\centering
 			 \includegraphics[scale=0.75]{"task 3 commit"}
 			 \caption{Adaugarea fisierelor la repozitoriu}
 			 \label{fig:clone_repo}
		\end{figure}
		
\item Crearea branch-urilor

Pentru a crea un branch este necesara utilizarea comenzii \textit{git branch <name>} Comanda \textit{git branch} ne afiseaza branch-urile. Pentru a crea un branch nou si a face switch la el se utilizeaza comanda \textit{git branch -b <name>}

\begin{figure}[h!]
			\centering
 			 \includegraphics[scale=0.75]{"task 4 create branch"}
 			 \caption{Crearea branch-urilor}
 			 \label{fig:clone_repo}
		\end{figure}

\item Crearea commit-urilor de pe fiecare branch

\end{itemize}

