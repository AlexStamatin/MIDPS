\section*{Concluzie}
\phantomsection


	In cadrul acestei lucrari de laborator am realizat un calculator cu interfata grafica pentru utilizator. In urma efectuarii lucrarii am obtinut deprinderi de utilizare a framework-ului Qt in limbajul C++. Am studiat procesul de instalare a add-on-urilor pentru Visual Studio si de creare a unei interfete grafice cu ajutorul Qt Designer.
	Am determinat un sir de avantaje al utilizarii framework-ului Qt
	\begin{itemize}
	
	\item Ofera toate instrumentele necesare pentru scrierea unei aplicatii
	\item Este valabil gratuit pentru dezvoltarea aplicatiilor open-source
	\item Este mai simplu in utilizare si performant ca alte produse
	
	\end{itemize}

	In acelasi timp are si anumite dezavantaje
	\begin{itemize}
	
	\item Nu sunt utilizati smart pointeri
	\item Nu sunt utilizate bibliotecile standard C++
	
	\end{itemize}

Interfata grafica este importanta pentru asigurearea unei interactiuni eficiente intre utilizatorul final si produsul soft. Elementele grafice utilizate trebuie sa fie intuitive iar scopul lor clar si bine-definit. Este preferabil ca interfata sa utilizeze la maxim cunstintele pe care utilizatorul deja le poseda in urma utilizarii altor aplicatii.

	Ca rezultat al efectuarii lucrarii de laborator am ajuns la concluzia ca este preferabil de separat modulul grafic si cel de baza intr-un program pentru a putea modifica un modul fara a aplica schimbari la celalalt. De asemenea am stabilit avantajele si importanta utilizarii unei interfete grafice si am obtinut deprinderi de creare a acesteia.

\clearpage