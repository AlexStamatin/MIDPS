\section{Efectuarea lucrarii de laborator}

\subsection{Sarcinile propuse}
\begin{itemize}
	\item Realizarea unei aplicatii mobile pentru testarea cunostintelor despre logo-urile aplicatiilor 
	\item Utilizarea mediului interactiv de dezvoltare Android Studio
\end{itemize}


\subsection{Analiza lucrarii de laborator}

Link la repozitoriu: https://github.com/AlexStamatin/MIDPS
	
	
	In cadrul acestei lucrari de laborator am elaborat o aplicatie pe Android ce verifica cunoasterea logo-urilor celor mai populare aplicatii pentru android. Pentru realizarea acestui scop a fost utilizat programul Android Studio. 

\begin{itemize}
	\item Interfata aplicatiei
	

	Interfata aplicatiei a fost realizata cu ajutorul instrumentelor 	din Android Studio. Logourile sunt prezentate sub forma de imagine si utilizatorul are posibilitatea de a selecta literele ce se contin in denumirea aplicatiei.
		\begin{figure}[h!]
			\centering
 			 \includegraphics[scale=1]{"Hangouts"}
 			 \caption{Interfata aplicatiei}
 			 \label{fig:Pagina principala}
		\end{figure}	
	\item Functionalitatea aplicatiei
	
	Utilizatorul are posibilitatea de a alege literele ce fac parte din denumirea aplicatiei careia ii apartine logoul si verifica daca alegerea a fost corecta.
	\begin{figure}[h!]
			\centering
 			 \includegraphics[scale=1]{"Choiceright"}
 			 \caption{Alegerea a fost facuta cu succes}
 			 \label{fig:Calcui}
		\end{figure}
		


	\begin{figure}[h!]
			\centering
 			 \includegraphics[scale=1]{"choicewrong"}
 			 \caption{Alegerea a fost facuta gresit}
 			 \label{fig:Calcui}
		\end{figure}


\end{itemize}
